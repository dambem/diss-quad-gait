\chapter{Introduction}
% Animal movement has been replicated

In nature, rhythmic movements such as walking, breathing or chewing are controlled by neural networks providing co-ordinated patterns for actions. These networks are more commonly referred to as Central Pattern Generators (CPGs) \citep{MarderEveBucher2001}. This dissertation will focus  on the use of CPGs in the creation of animal locomotion. One way to represent CPGs is through the use of coupled non-linear oscillators that generate rhythmic patterns, non-linear referring to the periodic signal of the oscillator not moving constantly at the same rate. A single non-linear oscillator can represent the rhythmic movement for a single limb of an animal. Gaits can  be replicated by creating coupling in-between oscillators. Coupling provides a method of replicating gaits by the creation of hard-wired links between oscillator values that create different rhythmic patterns, and so can replicate actions such as walking, trotting or running. 

Animal movement has been shown to follow certain patterns - many mammals share similar styles of walks, trots and other gaits. Dynamic Similarity is a hypothesis that claims two different mammals travelling with similar Froude Numbers (a value derived from simple measurements about an animal) \footnote{Froude Number may refer to other physical systems \citep{Pratt2008}. All mentions of a Froude Number in this dissertation refer to the Froude Value described in \cite{Alexander1983}} will use the same gaits \citep{Alexander1983}. Froude Number can be represented by equation \ref{froude:equation1}, where v represents velocity, g represents gravitational acceleration and h represents height at the hip.

\begin{equation}
F = v^2/gh
\label{froude:equation1}
\end{equation}

Although this hypothesis was only applied to living mammals, there have since been many examples of quadruped robots designed to move in a similar fashion to living mammals, from MiT's Cheetah \citep{Wensing2017} to  Boston Dynamic's Big Dog \citep{Raibert2008}. This leads to the hypothesis that a quadruped robot performing generated gaits will also need to adhere to the Dynamic Similarity hypothesis.

Despite the large number of robotic quadrupeds designed to perform like mammals, There exists little research on whether the concept of Dynamic Similarity would still apply to a robotic quadruped. This research aims to examine and compare the similarities and differences between a simulated robotic quadruped and information on animals found in the Dynamic Similarity Hypothesis.
% We i this through the comparison of a simple robot
% we could in turn estimate parameters such as speed and time period the animal should be travelling at. This could provide an important initial estimate on what the parameters of a given Central Pattern Generator, to maintain both balance, and reduce cost of travel \cite{Yong2005}. Currently, a large amount of literature uses parameters either from parameter tweaking, or estimates. 

\section{Aims and Objectives}
This dissertation will implement a CPG into a physics based simulation, through the use of a robot model based on a real counterpart. Although the Dynamic Similarity hypothesis has been applied to comparing human and dog gaits \citep{Gan2018a}, a direct comparison has not yet been made between quadruped robots and living mammals. 

An evaluation will be made as to whether these values are within the confines found in \cite{Alexander1983}. This will be done through calculating the Froude Number for a simulated quadruped and comparing it against Froude Number values found in the Dynamic Similarity Hypothesis.

This aims to find out if the Dynamic Similarity Hypothesis still holds for a simulated quadruped using coupled oscillators for its CPG. This  aims to showcase that the Dynamic Similarity hypothesis can be applied to real quadruped robots in the future. This aims to showcase a potential method of finding parameters for quadruped robots in future research, as well as a potential rubric for investigating robotic gaits.

% , as given the height of an arbitrary robot, we can in turn find the range of velocities that different gaits should be performing at.

\section{Overview}
This dissertation will begin with a literature survey describing in  detail the concept of Dynamic Similarity followed by an overview of gait generation methods. A justification will be made for why coupled Van der Pol oscillators where decided to be the method of gait generation followed by a detailed description of how this method was implemented. An evaluation will be made for which robot environment is most appropriate for creating a comparison. 

Testing criteria used to find appropriate ranges of parameters for testing will be described, along with a description of the testing environment for the final findings. The model made will be compared to values found in the Dynamic Similarity hypothesis, and show whether gait generation using a coupled oscillator as a Central Pattern Generator produces gaits similar to that seen in actual mammals. There will be a discussion of the findings, detailing the successes and failures of the project.This will be followed with  further work that could be done with the model created.

\section{Relationship Between Project and Degree Program}
The work done in this dissertation provide close links with modules such as Natural Systems and Adaptive Intelligence. This has proven advantageous, in that both of these modules contain assignments based on writing academic reports. Additionally, the work on derivatives in natural systems provided a more in depth understanding of how the van der pol oscillator worked, giving a clearer understanding of derivative solutions and equations, and the implementation of them in Python. However, a large amount of the work done in this dissertation has not been part of the computer science curriculum, especially proper statistics design and T-tests for similarity, this required a large amount of research on appropriate tests to run.
