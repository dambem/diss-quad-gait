\chapter*{\Large \center Abstract}

% Guidance of how to write an abstract/summary provided by Nature: https://cbs.umn.edu/sites/cbs.umn.edu/files/public/downloads/Annotated_Nature_abstract.pdf

The Dynamic Similarity hypothesis states that different mammals will exhibit dynamically similar movement when travelling at a velocity that grants them equal Froude Numbers, a calculation based on measurements of an animal's height, which indicates the gait an animal will be performing. Gaits can be generated for a quadruped robot through the replication of Central Pattern Generators (biological neural circuits in animals that produce rhythmic outputs) using groups of coupled non-linear oscillators producing periodic oscillatory movement, with coupling performed through the use of neuronal links between oscillators. Other methods used to replicate quadruped movement tend to either be limited to common species due to difficulties in collecting training data for wild animals or if done through other mathematical methods, do not compare themselves directly to their real life counterparts. Here we show that simulated gaits using coupled Van der Pol oscillators adhere to the dynamic similarity hypothesis, with over 70\%  of successful walking gaits exhibiting ranges of Froude number found in living mammals. These results demonstrate that although simulated gaits can reach larger Froude numbers (and hence faster velocities) than evolutionary values, Froude number can be used as a valid guideline for successful gait generation. This research provides a starting point for investigating and comparing simulated gaits to living mammals. Furthermore, this is the first open-source implementation of coupled oscillators as Central Pattern Generators, and as such could be used for more sophisticated research into the design of quadruped gaits in the future. 

% We will show how natural gaits can be generated for a quadruped robot through replication of a central pattern generator by the use of Coupled Van der Pol Oscillators and compare these values to those found for real mammals in the Dynamic Similarity Hypothesis. By investigating effects on In addition to this, we find that combinations of parameters who's values reach above Froude numbers seen in evolutionary principles produce far more variation, and are more likely to produce unstable and ineffective movement than combinations in which more values lie between acceptable values.

% These results show that provide a starting point for the development of realistic mammal gaits, and a method of comparing robotic gaits to real counterparts. This could have potential uses in the initial stages of quadruped gait design, as  `